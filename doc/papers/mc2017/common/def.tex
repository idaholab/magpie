%
%=================================================================================================
% new commands
% +++++++++++++++++++++++++++++++++++++++++++++++++++++++++++++++++++++++++++++++++++++++++++++++++
\newcommand{\nc}{\newcommand}
%
% Ways of grouping things
%
\newcommand{\bracket}[1]{\left[ #1 \right]}
\newcommand{\bracet}[1]{\left\{ #1 \right\}}
\newcommand{\fn}[1]{\left( #1 \right)}
\newcommand{\ave}[1]{\left\langle #1 \right\rangle}
%
% Derivative forms
%
\newcommand{\dx}[1]{\,d#1}
\newcommand{\dxdy}[2]{\frac{\partial #1}{\partial #2}}
\newcommand{\dxdt}[1]{\frac{\partial #1}{\partial t}}
\newcommand{\dxdz}[1]{\frac{\partial #1}{\partial z}}
\newcommand{\dfdt}[1]{\frac{\partial}{\partial t} \fn{#1}}
\newcommand{\dfdz}[1]{\frac{\partial}{\partial z} \fn{#1}}
\newcommand{\ddt}[1]{\frac{\partial}{\partial t} #1}
\newcommand{\ddz}[1]{\frac{\partial}{\partial z} #1}
\newcommand{\dd}[2]{\frac{\partial}{\partial #1} #2}
\newcommand{\ddx}[1]{\frac{\partial}{\partial x} #1}
\newcommand{\ddy}[1]{\frac{\partial}{\partial y} #1}
%
% Vector forms
%
%\renewcommand{\vec}[1]{\ensuremath{\stackrel{\rightarrow}{#1}}}
%\renewcommand{\div}{\ensuremath{\vec{\nabla} \cdot}}
%\newcommand{\grad}{\ensuremath{\vec{\nabla}}}

\renewcommand{\div}{\vec{\nabla}\! \cdot \!}
\newcommand{\grad}{\vec{\nabla}}
\newcommand{\oa}[1]{\fn{\frac{1}{3}\hat{\Omega}\!\cdot\!\overrightarrow{A_{#1}}}}

%
% Equation beginnings and endings
%
\newcommand{\bea}{\begin{eqnarray}}
\newcommand{\eea}{\end{eqnarray}}
\newcommand{\be}{\begin{equation}}
\newcommand{\ee}{\end{equation}}
\newcommand{\beas}{\begin{eqnarray*}}
\newcommand{\eeas}{\end{eqnarray*}}
\newcommand{\bdm}{\begin{displaymath}}
\newcommand{\edm}{\end{displaymath}}
%
% Equation punctuation
%
\newcommand{\pec}{\hspace{0.25in},}
\newcommand{\pep}{\hspace{0.25in}.}
\newcommand{\pev}{\hspace{0.25in}}
%
% Equation labels and references, figure references, table references
%
\newcommand{\LEQ}[1]{\label{eq:#1}}
\newcommand{\EQ}[1]{Eq.~(\ref{eq:#1})}
\newcommand{\EQS}[1]{Eqs.~(\ref{eq:#1})}
\newcommand{\REQ}[1]{\ref{eq:#1}}
\newcommand{\LFI}[1]{\label{fi:#1}}
\newcommand{\FI}[1]{Fig.~\ref{fi:#1}}
\newcommand{\RFI}[1]{\ref{fi:#1}}
\newcommand{\LTA}[1]{\label{ta:#1}}
\newcommand{\TA}[1]{Table~\ref{ta:#1}}
\newcommand{\RTA}[1]{\ref{ta:#1}}

%
% List beginnings and endings
%
\newcommand{\bl}{\bss\begin{itemize}}
\newcommand{\el}{\vspace{-.5\baselineskip}\end{itemize}\ess}
\newcommand{\benu}{\bss\begin{enumerate}}
\newcommand{\eenu}{\vspace{-.5\baselineskip}\end{enumerate}\ess}
%
% Figure and table beginnings and endings
%
\newcommand{\bfg}{\begin{figure}}
\newcommand{\efg}{\end{figure}}
\newcommand{\bt}{\begin{table}}
\newcommand{\et}{\end{table}}
%
% Tabular and center beginnings and endings
%
\newcommand{\bc}{\begin{center}}
\newcommand{\ec}{\end{center}}
\newcommand{\btb}{\begin{center}\begin{tabular}}
\newcommand{\etb}{\end{tabular}\end{center}}
%
% Single space command
%
%\newcommand{\bss}{\begin{singlespace}}
%\newcommand{\ess}{\end{singlespace}}
\newcommand{\bss}{\singlespacing}
\newcommand{\ess}{\doublespacing}
%
%---New environment "arbspace". (modeled after singlespace environment
%                                in Doublespace.sty)
%   The baselinestretch only takes effect at a size change, so do one.
%
\def\arbspace#1{\def\baselinestretch{#1}\@normalsize}
\def\endarbspace{}
\newcommand{\bas}{\begin{arbspace}}
\newcommand{\eas}{\end{arbspace}}
%
% An explanation for a function
%
\newcommand{\explain}[1]{\mbox{\hspace{2em} #1}}
%
% Quick commands for symbols
%
\newcommand{\half}{\frac{1}{2}}
\newcommand{\third}{\frac{1}{3}}
\newcommand{\twothird}{\frac{2}{3}}
\newcommand{\fourth}{\frac{1}{4}}
\newcommand{\mdot}{\dot{m}}
\newcommand{\ten}[1]{\times 10^{#1}\,}
\newcommand{\cL}{{\cal L}}
\newcommand{\cD}{{\cal D}}
\newcommand{\cF}{{\cal F}}
\newcommand{\cE}{{\cal E}}
\renewcommand{\Re}{\mbox{Re}}
\newcommand{\Ma}{\mbox{Ma}}
%
% Inclusion of Graphics Data
%
%\input{psfig}
%\psfiginit
%
% More Quick Commands
%
\newcommand{\bi}{\begin{itemize}}
\newcommand{\ei}{\end{itemize}}
\newcommand{\ben}{\begin{enumerate}}
\newcommand{\een}{\end{enumerate}}
\newcommand{\dxi}{\Delta x_i}
\newcommand{\dyj}{\Delta y_j}
\newcommand{\ts}[1]{\textstyle #1}


\newcommand{\bu}{\boldsymbol{u}}
\newcommand{\ber}{\boldsymbol{e}}
\newcommand{\br}{\boldsymbol{r}}
\newcommand{\bo}{\boldsymbol{\Omega}}

\newcommand{\bn}{\boldsymbol{\nabla}}

% DGFEM commands
\newcommand{\jmp}[1]{[\![#1]\!]}                     % jump
\newcommand{\mvl}[1]{\{\!\!\{#1\}\!\!\}}             % mean value
\newcommand{\jmpa}[1]{[\![\![#1]\!]\!]}              % jump


\newcommand{\boxedeqn}[1]{%
  \[\fbox{%
      \addtolength{\linewidth}{-2\fboxsep}%
      \addtolength{\linewidth}{-2\fboxrule}%
      \begin{minipage}{\linewidth}%
      \begin{equation}#1\end{equation}%
      \end{minipage}%
    }\]%
}
\newcommand{\mboxed}[1]{\boxed{\phantom{#1}}}
\newcommand{\ud}{\,\mathrm{d}}

% keff
\newcommand{\keff}{\ensuremath{k_{\textit{eff}}}\xspace}

% margin par
\newcommand{\mt}[1]{\marginpar{ {\footnotesize #1} }}

% shortcut for aposterio in italics
\newcommand{\apost}{\textit{a posteriori\xspace}}
\newcommand{\Apost}{\textit{A posteriori}\xspace}

% shortcut for multi-group
\newcommand{\mg}{multigroup\xspace}
\newcommand{\Mg}{Multigroup\xspace}
\newcommand{\ho}{higher-order\xspace}
\newcommand{\Ho}{Higher-order\xspace}
\newcommand{\HO}{Higher-Order\xspace}
\newcommand{\HObig}{HIGHER-ORDER\xspace}
\newcommand{\Mgbig}{MULTIGROUP\xspace}
\newcommand{\sn}{$S_N$\xspace}
\newcommand{\pn}{$P_N$\xspace}

% shortcut for Rattlesnake
\newcommand{\rattlesnake}{Rattlesnake}

% shortcut for domain notation
\newcommand{\D}{\mathcal{D}}
\newcommand{\Sp}{\mathcal{S}}

% shortcut for xuthus
\newcommand{\psc}[1]{{\sc {#1}}}
\newcommand{\xuthus}{\psc{xuthus}\xspace}

% vector shortcuts
\newcommand{\vo}{\vec{\Omega}}
\newcommand{\vr}{\vec{r}}
\newcommand{\vn}{\vec{n}}
\newcommand{\vnk}{\vec{\mathbf{n}}}

% extra space
\newcommand{\qq}{\quad\quad}

% sign function
\DeclareMathOperator{\sgn}{sgn}


\makeatletter
\newcommand{\rmnum}[1]{\romannumeral #1}
\newcommand{\Rmnum}[1]{\expandafter\@slowromancap\romannumeral #1@}
\makeatother

\newcommand{\ensuretext}[1]{\ensuremath{\text{#1}}}
\newcommand{\Rmnumb}[1]{\ensuretext{\Rmnum{#1}}}

% common reference commands
\newcommand{\eqt}[1]{Eq.~(\ref{#1})}                     % equation
\newcommand{\fig}[1]{Fig.~\ref{#1}}                      % figure
\newcommand{\tbl}[1]{Table~\ref{#1}}                     % table
\newcommand{\app}[1]{Appendix~\ref{#1}}                  % appendix


% for mathematica notebook
\newcommand{\IndentingNewLine}{ \\ }


\newcommand{\rhs}{right-hand-side\xspace}
\newcommand{\clearemptydoublepage}{\newpage{\pagestyle{empty}\cleardoublepage}}

\newenvironment{myverbatim}%            To change the pseudocode font
{\par\noindent%
 \rule[0pt]{\linewidth}{0.2pt}
 \vspace*{-9pt}
 \linespread{0.0}\small\verbatim}%
{\rule[-5pt]{\linewidth}{0.2pt}\endverbatim}

\newenvironment{myverbatim1}%            To change the pseudocode font
{\par\noindent%
 \rule[0pt]{\linewidth}{0.2pt}
 \vspace*{-9pt}
 \linespread{1.0}\scriptsize\verbatim}%
{\rule[-5pt]{\linewidth}{0.2pt}\endverbatim}

\newcommand{\theHalgorithm}{\arabic{algorithm}} % remove the error of algorithm+hyperref

%\hypersetup{
%    bookmarks=true,         % show bookmarks bar?
%    unicode=false,          % non-Latin characters in Acrobat's bookmarks
%    pdftoolbar=true,        % show Acrobat's toolbar?
%    pdfmenubar=true,        % show Acrobat's menu?
%    pdffitwindow=false,     % window fit to page when opened
%    pdfstartview={FitH},    % fits the width of the page to the window
%    pdftitle={Dissertation},    % title
%    pdfauthor={Yaqi Wang},     % author
%    pdfsubject={Transport AMR},   % subject of the document
%    pdfcreator={Yaqi Wang},   % creator of the document
%    pdfproducer={Yaqi Wang}, % producer of the document
%    pdfkeywords={Transport, AMR}, % list of keywords
%    pdfnewwindow=true,      % links in new window
%    colorlinks=false,       % false: boxed links; true: colored links
%    linkcolor=red,          % color of internal links
%    citecolor=green,        % color of links to bibliography
%    filecolor=magenta,      % color of file links
%    urlcolor=cyan           % color of external links
%}

% prepare generating nomenclature and change default options
%\makenomenclature
%\renewcommand{\nomname}{NOMENCLATURE}
%\RequirePackage{ifthen}
%\renewcommand{\nomgroup}[1]{%
%\item[]\hspace*{-\leftmargin}%
%%\rule[2pt]{0.45\linewidth}{1pt}%
%%\hfill
%\ifthenelse{\equal{#1}{A}}{\textbf{Abbreviations}}{%
%\ifthenelse{\equal{#1}{S}}{\textbf{Symbols}}{
%\ifthenelse{\equal{#1}{U}}{\textbf{Superscripts}}{
%\ifthenelse{\equal{#1}{V}}{\textbf{Subscripts}}{}}}}
%%\hfill
%%\rule[2pt]{0.45\linewidth}{1pt}
%}

% a new environment for splitting a long algorithm
\makeatletter
\newenvironment{breakalgo}[2][alg:\thealgorithm]{%
  \def\@fs@cfont{\bfseries}%
  \let\@fs@capt\relax%
  \par\noindent%
  \medskip%
  \rule{\linewidth}{.8pt}%
  \vspace{-20pt}%
  %\par\noindent
  \captionof{algorithm}{#2}\label{#1}%
  \vspace{-1.2\baselineskip}%
%  \noindent\rule{\linewidth}{.4pt}%
  \vspace{8pt}%
  \noindent\rule{\linewidth}{.4pt}%
  \vspace{-1.3\baselineskip}%
}{%
  \vspace{-.75\baselineskip}%
  \par\noindent%
  \rule{\linewidth}{.4pt}%
  \medskip%
}
\makeatother

